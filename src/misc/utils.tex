% utils.tex
% Meta Information
\newcommand{\where}{}
\newcommand{\recorder}{}
\newcommand{\chair}{}
\newcommand{\meetingDate}{}
\newcommand{\meetingStartTime}{}
\newcommand{\meetingEndTime}{}
\newcommand{\meetingNextTime}{}
\newcommand{\meetingNextDate}{}
\newcommand{\meetingNext}{\meetingNextDate, \meetingNextTime}
\newcommand{\meetingAttendence}{}
\newcommand{\meetingAttendenceMax}{}
\newcommand{\approvalTO}{}
\newcommand{\approvalProtocol}{}

% set Meta Information
\newcommand{\setWhere}      [1]{\renewcommand{\where}{#1}}
\newcommand{\setDate}       [1]{\renewcommand{\meetingDate}{#1}}
\newcommand{\setRecorder}   [1]{\renewcommand{\recorder}{#1}}
\newcommand{\setChair}      [1]{\renewcommand{\chair}{#1}}
\newcommand{\setStartTime}  [1]{\renewcommand{\meetingStartTime}{#1}}
\newcommand{\setEndTime}    [1]{\renewcommand{\meetingEndTime}{#1}}
\newcommand{\setNextTime}   [1]{\renewcommand{\meetingNextTime}{#1}}
\newcommand{\setNextDate}   [1]{\renewcommand{\meetingNextDate}{#1}}
\newcommand{\setAttendence} [1]{\renewcommand{\meetingAttendence}{#1}}
\newcommand{\setAttendenceMax} [1]{\renewcommand{\meetingAttendenceMax}{#1}}
\newcommand{\setApprovalTO} [1]{\renewcommand{\approvalTO}{#1}}
\newcommand{\setApprovalProtocol} [1]{\renewcommand{\approvalProtocol}{#1}}

% Helper Functions

% \quorum
% Check if the meeting can make decisions
% If more than 50% of the members are present, the meeting is quorate
%
% Usage:
% \quorum
\newcommand{\quorum}{
  % Calculate the quorum
  \pgfmathsetmacro{\quorum}{\meetingAttendence/\meetingAttendenceMax}
  % Check if the quorum is greater than 0.5
  \pgfmathparse{ifthenelse(\quorum > 0.5, "Beschlussfähig", "Nicht Beschlussfähig")}
  \pgfmathresult
}

% \addTopic
% Add a new topic to the protocol with the given title
%
% Usage:
% \addTopic{TOPIC TITLE}
\newcounter{topiccounter}
\newcommand{\addTopic}[1]{
    \stepcounter{topiccounter}
    \addsec{TOP \arabic{topiccounter} - #1}
}

% \makeTitle
% Print the Title of the protocol
%
% Usage:
% \makeTitle
\newcommand{\makeTitle}{
  \section*{\centering\huge Fachschaftsratssitzung am \meetingDate}
}

% \makePreamble
% Print the Preamble of the protocol
%
% Usage:
% \makePreamble
\newcommand{\makePreamble}{
    \begin{multicols}{2}
        Datum: \meetingDate \\
        Ort: \where \\
        Sitzungsleitung: \chair \\
        Offizieller Start: \meetingStartTime \\
        Offizielles Ende: \meetingEndTime \\
        Nächste Sitzung: \meetingNext
    \end{multicols}
}

% \makePostamble
% Print the Postamble of the protocol
%
% Usage:
% \makePostamble
\newcommand{\makePostamble}{
    \vspace{2.5cm}
    \begin{spacing}{0.3}
    \noindent\rule[1ex]{\textwidth}{1pt}
    \begin{multicols}{2}
        Unterschrift, Datum \\
        Protokollant*in \\
        Unterschrift, Datum \\
        Sprecher*in
    \end{multicols}
    \end{spacing}
}

% \makeResolution
% Print a resolution result with the given values
%
% Usage:
% \makeResolution{TOPIC}{COUNT_YES}{COUNT_NO}{COUNT_ABSTAIN}
\newcommand{\makeResolution}[4]{
  \textbf{Beschluss:} #1 ? \textbf{(#2|#3|#4)}
}

% \makeQuote
% Print a quote from a attendee
%
% Usage:
% \makeQuote{NAME}{QUOTE}
\newcommand{\makeQuote}[2]{
  \textbf{#1:} \textsf{#2}
}

% \addAttendee
% Add an attendee to the protocol
%
% Usage:
% \addAttendee{NAME}{TYPE}{ATTENDING}{COMMENT}
\newcommand{\addAttendee}[4]{%
    \IfStrEq{#2}{elected}{%
        \stepcounter{elected}%
        \addElectedAttendee{#1}{#3}{#4}%
        \ifthenelse{#3=1}{\stepcounter{electedattending}}{}%
    }{}%
    \IfStrEq{#2}{substitute}{%
        \stepcounter{substitute}%
        \addSubstituteAttendee{#1}{#3}{#4}%
    }{}%
    \IfStrEq{#2}{volunteer}{%
        \stepcounter{volunteer}%
        \addVolunteerAttendee{#1}{#3}{#4}%
    }{}%
    \IfStrEq{#2}{guest}{%
        \stepcounter{guest}%
        \addGuestAttendee{#1}{#3}{#4}%
    }{}%
}

% Command to make attendance tables
% 
% Usage:
% \makeAttendance
\newcommand{\makeAttendance}{%
    \section*{Anwesendheit}

    \ifthenelse{\value{elected} > 0}{%
	      \begin{tabularx}{\textwidth}{|p{4cm}|c|c|X|}
        \hline
        \textbf{Name} & \textbf{Anwesend} & \textbf{Abwesend} & \textbf{Bemerkung} \\
        \hline
        \the\electedRows
        \end{tabularx}
        \vspace{1em}
        \setAttendence{\theelectedattending}
        \setAttendenceMax{\theelected}
    }{}

    \ifthenelse{\value{substitute} > 0}{%
        \begin{tabularx}{\textwidth}{|p{4cm}|c|c|X|}
        \hline
        \textbf{Name} & \textbf{Anwesend} & \textbf{Abwesend} & \textbf{Bemerkung} \\
        \hline
        \the\substituteRows
        \end{tabularx}
    }{}
    
    \ifthenelse{\value{volunteer} > 0}{%
        \begin{tabularx}{\textwidth}{|p{4cm}|c|c|X|}
        \hline
        \textbf{Name} & \textbf{Anwesend} & \textbf{Abwesend} & \textbf{Bemerkung} \\
        \hline
        \the\volunteerRows
        \end{tabularx}
    }{}

    \ifthenelse{\value{guest} > 0}{%
        \begin{tabularx}{\textwidth}{|p{4cm}|c|c|X|}
        \hline
        \textbf{Name} & \textbf{Anwesend} & \textbf{Abwesend} & \textbf{Bemerkung} \\
        \hline
        \the\guestRows
        \end{tabularx}
    }{}
}

% Command to add an attendee to the elected table
\newcommand{\addElectedAttendee}[3]{%
    \electedRows=\expandafter{\the\electedRows #1 & \ifthenelse{#2=1}{\checkmark}{} & \ifthenelse{#2=1}{}{\checkmark} & #3 \\ \hline}%
}

% Command to add an attendee to the substitute table
\newcommand{\addSubstituteAttendee}[3]{%
    \substituteRows=\expandafter{\the\substituteRows #1 & \ifthenelse{#2=1}{\checkmark}{} & \ifthenelse{#2=1}{}{\checkmark} & #3 \\ \hline}%
}

% Command to add an attendee to the volunteer table
\newcommand{\addVolunteerAttendee}[3]{%
    \volunteerRows=\expandafter{\the\volunteerRows #1 & \ifthenelse{#2=1}{\checkmark}{} & \ifthenelse{#2=1}{}{\checkmark} & #3 \\ \hline}%
}
% Command to add an attendee to the guest table
\newcommand{\addGuestAttendee}[3]{%
    \guestRows=\expandafter{\the\guestRows #1 & \ifthenelse{#2=1}{\checkmark}{} & \ifthenelse{#2=1}{}{\checkmark} & #3 \\ \hline}%
}
