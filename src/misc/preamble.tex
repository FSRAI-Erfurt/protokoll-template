\usepackage[utf8]{inputenc}
\usepackage[T1]{fontenc}
\usepackage[ngerman]{babel}
\usepackage[left=3cm, right=3cm, top=2.5cm, bottom=2.5cm]{geometry}
\usepackage{multicol}
\usepackage{tabularx}
\usepackage{helvet}
\renewcommand{\familydefault}{\sfdefault}
\usepackage[onehalfspacing]{setspace}
\usepackage{cancel}
\usepackage[babel,german=quotes]{csquotes}
\usepackage[ngerman]{translator}
\usepackage{booktabs}
\usepackage{array}
\usepackage{xstring}
\usepackage{booktabs}
\usepackage{amssymb}
\usepackage[ngerman, num]{isodate}
\monthyearsepgerman{\,}{\,}
\usepackage{tabulary}
\usepackage{ifthen}
\usepackage[automark,plainheadsepline,autooneside]{scrlayer-scrpage}
\usepackage{lastpage}
\usepackage{graphicx}
\pagestyle{scrheadings}
\clearpairofpagestyles

% Add these lines for custom page numbering format
\cfoot[\pagemark]{Seite \thepage\ / \pageref{LastPage}}
\renewcommand{\pagemark}{\thepage}

% Your document content goes here

\renewcommand*{\headfont}{\upshape\sffamily\scriptsize}
\renewcommand*{\footfont}{\normalfont\sffamily\small}

\usepackage{xcolor,listings}
\definecolor{comment}{rgb}{.15,.4,.15}     	% hellgruen

% tableofcontents should use a heading: "Tagesordnung"
\renewcaptionname{ngerman}{\contentsname}{Tagesordnung}

% Hier kann jetzt alles für die verwendete Sprache eingestellt werden
% Der lstset-Befehl ermöglicht haufenweise Einstellungen zur Formatierung

\lstset{language=C++,                            % hier Sprache einstellen
	basicstyle={\small} ,                        % Schriftgröße
	keywordstyle=\color{blue!80!black!100},      % Farbe der keywords
	identifierstyle=,                            % Bezeichnerstyle, hier leer
	commentstyle=\color{green!50!black!100},     % Farbe der Kommentare
	stringstyle=\ttfamily,                       % Aussehen der Strings
	breaklines=true,                             % Automatische Zeilenumbrüche
	numbers=left,                                % Zeilennummerierung links
	numberstyle=\small,                          % Größe der Zeilennummerierung
	frame=single,                                % einfacher Rahmen
	backgroundcolor=\color{blue!3},              % Hintergrundfarbe
	caption={Code},                              % Standardüberschrift
	captionpos=t,                                % Überschift oben (top)
	showspaces=false,							 % Leerzeichen nicht anzeigen
	showstringspaces=false,						 % Stringleerzeichen nicht anzeigen
	literate=
	{Ä}{{\"A}}1
	{Ö}{{\"O}}1
	{Ü}{{\"U}}1
	{ß}{{\ss}}1
	{ä}{{\"a}}1
	{ö}{{\"o}}1
	{ü}{{\"u}}1
	{~}{{\textasciitilde}}1
}
